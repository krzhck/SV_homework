%% compile with xelatex
\documentclass[11pt,a4paper]{article}

\usepackage[UTF8]{ctex}
\usepackage{homework}

\title{作业 4}
\duedate{May 8, 2022}
% TODO your name and ID
\studentname{周雨豪}
\studentid{2018013399}

\usepackage{tikz}
\usepackage{listings}
\usetikzlibrary{automata,shapes,positioning,arrows}
\lstdefinelanguage{pimp}{
  morekeywords={assume, while, and, do, od, if, fi, then, else, skip, int, not, return, assert, print, procedure, true, false, requires, ensures}, 
  sensitive=false, 
  morecomment=[l][\color{blue}]{//}, 
  morecomment=[s][\color{purple}]{/*}{*/}, 
  morestring=[b]", 
}

\lstset{
  basicstyle=\ttfamily,
  language=pimp,
  escapechar=|,
  literate=%
  {~}{{$\neg$}}1        % \neg
  {-}{{-}}1             % prevent ``-'' being replaced by math minus
  {delta}{{$\Delta$}}1  % \Delta
}
%% logical symbols
% \land     /\
% \lor      \/
% \lnor     (negation)
% \to       ->
% \lequiv   <->
% \models   |=
\newcommand{\lequiv}{\leftrightarrow}
\newcommand{\func}[1]{\textbf{#1}}

% 模型:$\mathcal{M} = (\mathcal{D}, \mathcal{I})$
% 论域:$\mathcal{D}$
% 解释:$\mathcal{I}$
\newcommand{\model}{\mathcal{M}}
\newcommand{\domain}{\mathcal{D}}
\newcommand{\interpretation}{\mathcal{I}}

% 变量赋值:$\assignment: \varset \to \domain$
\let\mapping\rho
\let\assignment\rho
\let\env\mapping

% 语义:$|[\cdot|]$
\newcommand{\sem}[1]{|[#1|]}                        % 程序语义 / 无下标的语义
\newcommand{\plsem}[1]{|[#1|]_{\assignment}}        % 命题逻辑语义
\newcommand{\fosem}[1]{|[#1|]_{\model,\assignment}} % 一阶逻辑语义

% 语义蕴涵和语义等价: $\simply, \sequiv$
\let\simply\entails
\let\sequiv\Leftrightarrow

\begin{document}

\maketitle

\textit{在开始完成作业前,请仔细阅读以下说明:}
\begin{itemize}
    \item 我们提供作业的\LaTeX 源码,你可以在其中直接填充你的答案并编译PDF(请使用xelatex)。
    当然,你也可以使用别的方式完成作业(例如撰写纸质作业后扫描到PDF文件之中)。
    但是请注意,最终的提交一定只是PDF文件。提交时请务必\emph{再次核对},防止提交错误。
    \item 在你的作业中,请务必填写你的\emph{姓名}和\emph{学号},并检查是否有题目遗漏。请重点注意每次作业的\emph{截止时间}。
    截止时间之后你仍可以联系助教补交作业,但是我们会按照如下公式进行分数的折扣:
    \begin{align*}
        \text{作业分数} = \min\left(\text{实际分}, \text{满分}\times\left(1 - 10\%\times\min\left(\lceil\text{迟交周数}\rceil, 10\right)\right)\right).
    \end{align*}
    \item 本次作业为独立作业,\emph{禁止}抄袭等一切不诚信行为。作业中,如果涉及参考资料,请\emph{引用}注明。
\end{itemize}

%% problem begins

\problem{谓词变换}

\subproblem 计算下列最弱前置条件。
\begin{itemize}
    \item $wlp(b[m]:=b[n];b[n]:=t, b[m] < b[n])$
    \item $wlp(\textbf{if}\ y > 2\ \textbf{then}\ x:= y - 5\ \textbf{else}\ x:= -y, x \ge 0)$
\end{itemize}
\begin{solution}
    % TODO
    \begin{align*}
	    &wlp(b[m]:=b[n];b[n]:=t, b[m] < b[n]) \\
	    =\ &wlp(b[m]:=b[n],wlp(b[n]:=t,b[m]<b[n])) \\
	    =\ &wlp(b[m]:=b[n],(b[m]<b[n])[b\rightarrow b\langle{n\triangleleft t}\rangle]) \\
	    =\ &wlp(b[m]:=b[n],(b\langle{n\triangleleft t}\rangle[m]<b\langle{n\triangleleft t}\rangle[n])) \\
	    =\ &wlp(b[m]:=b[n],b[m]<t) \\
	    =\ &(b[m]<t)[b\rightarrow b\langle{m\triangleleft b[n]}\rangle] \\
	    =\ &(b\langle{m\triangleleft b[n]}\rangle[m]<t) \\
	    =\ &b[n]<t \\
    	&wlp(\textbf{if}\ y > 2\ \textbf{then}\ x:= y - 5\ \textbf{else}\ x:= -y, x \ge 0) \\
	   	=\ &(y>2\rightarrow wlp(x:=y-5,x\geq 0))\land(\neg(y>2)\rightarrow wlp(x:=-y,x\geq 0)) \\
	   	=\ &(y>2\rightarrow y-5\geq 0)\land(\neg(y>2)\rightarrow -y\geq 0) \\
	   	=\ &(y\leq 2 \lor y\geq 5)\land(y>2\lor y\leq 0) \\
	   	=\ &y\leq 0\lor y\geq 5
    \end{align*}
\end{solution}

\subproblem 利用最弱前置条件推导,证明下列程序属性的正确性。
\begin{lstlisting}
// {true}
n := 0;
x := 0;
while (r |$\neq$| 0) {
  n := n + 1;
  x := x + 2 |$\times$| n - 1;
  r := r - 1;
}
//{x = n |\color{blue}$\times$| n}
\end{lstlisting}

\textit{提示:考虑使用循环不变式:$x = n \times n$。}
\begin{solution}
\begin{lstlisting}
// {0 = 0 |\color{blue}$\times$| 0}
n := 0;
// {0 = n |\color{blue}$\times$| n}
x := 0;
// {x = n |\color{blue}$\times$| n}
while (r |$\neq$| 0) {...}
// {x = n |\color{blue}$\times$| n}
\end{lstlisting}
	\begin{align*}
		\text{前置条件}&\ \varphi:\ true \\
		\text{后置条件}&\ \psi:\ x=n\times n \land r=0 \\
		\text{循环不变式}&\ I:\ x=n\times n \\
		\text{验证条件}&:\\
		&1.\ I\land \neg(r\neq 0)\models \psi \\
		&2.\ I\land r\neq 0 \models wlp(n:=n+1;x:=x+2\times n-1;r:=r-1,I) \\
		&3.\ true\models 0=0\times 0 \\
		\text{验证条件}&\text{1、3均成立,对于验证条件2:}\\
		&wlp(n:=n+1;x:=x+2\times n-1;r:=r-1,I)\\
		=\ &wlp(n:=n+1,wlp(x:=x+2\times n-1,wlp(r:=r-1,I))) \\
		=\ &wlp(n:=n+1,wlp(x:=x+2\times n-1,x=n\times n)) \\
		=\ &wlp(n:=n+1,x+2\times n-1=n\times n) \\
		=\ &(x+2\times(n+1)-1=(n+1)\times(n+1)) \\
		=\ &(x=n\times n)=I \\
		\text{由此验证}&\text{条件2成立,示例程序满足给定规约。}
	\end{align*}
\end{solution}
\newpage
\problem{基本路径}
\subproblem 请写出过程Proc\_A的\emph{所有}基本路径。
\begin{lstlisting}
/* requires x > 0;
   ensures rv = 0; */
procedure Proc_M(x);

/* requires y > 0;
   ensures rv |\color{purple}$\ge$| 0; */
procedure Proc_A(y) {
    if (y > 10)
    {
        v := Proc_M(y);
        assert(v |$\ge$| 0);
        return v;
    }
    while(y > 0)
    /* invariant: y |\color{purple}$\ge$| 0 */
    {
        t := y;
        while(t > 0)
        /* invariant: t |\color{purple}$\ge$| 0 |\color{purple}$\wedge$| y |\color{purple}$\ge$| t */
        {
            t := t - 1;
        }
        y := y - 1;
    }
    return 0;
}

\end{lstlisting}
\begin{solution}
    % TODO
    \begin{align*}
    	\text{第一条基本路径}& \\
    	&\{y>0\} \\
    	&\textbf{assume}\ y>10; \\
    	&\{y>0\} \\
    	\text{第二条基本路径}& \\
    	&\{y>0\} \\
    	&\textbf{assume}\ y>10; \\
    	&\textbf{assume}\ v_M=0 \\
    	&v:=v_M \\
    	&\textbf{assume}\ v\geq 0; \\
    	&rv:=v \\
    	&\{rv\geq0\} \\
    	\text{第三条基本路径}& \\
    	&\{y>0\} \\
    	&\textbf{assume}\ y>10; \\
    	&\textbf{assume}\ v_M=0 \\
    	&v:=v_M \\
    	&\textbf{assume}\ v < 0; \\
    	&\{\text{错误状态}\} \\
    	\text{第四条基本路径}& \\
    	&\{y>0\} \\
    	&\textbf{assume}\ y\leq 10;\\
    	&\{y\geq 0\}\\
    	\text{第五条基本路径}& \\
    	&\{y\geq 0\} \\
    	&\textbf{assume}\ y>0; \\
    	&t:=y \\
    	&\{t \geq 0 \land y \geq t\} \\
    	\text{第六条基本路径}& \\
    	&\{t \geq 0 \land y \geq t\} \\
    	&\textbf{assume}\ t>0; \\
    	&t:=t-1; \\
    	&\{t \geq 0 \land y \geq t\} \\
    \end{align*}
    	
   	\begin{align*}
    	\text{第七条基本路径}& \\
    	&\{t \geq 0 \land y \geq t\} \\
    	&\textbf{assume}\ t\leq0; \\
    	&y:=y-1; \\
    	&\{y\geq 0\} \\
    	\text{第八条基本路径}& \\
    	&\{y\geq 0\} \\
    	&\textbf{assume}\ y\leq 0; \\
    	&rv:=0; \\
    	&\{rv\geq 0\} \\
    \end{align*}
\end{solution}


\end{document}
