%% compile with xelatex
\documentclass[11pt,a4paper]{article}

\usepackage[UTF8]{ctex}
\usepackage{homework}

\title{作业 3}
\duedate{Apr 17, 2022}
% TODO your name and ID
\studentname{你的姓名}
\studentid{你的学号}

\usepackage{tikz}
\usetikzlibrary{automata,shapes,positioning,arrows}

%% logical symbols
% \land     /\
% \lor      \/
% \lnor     (negation)
% \to       ->
% \lequiv   <->
% \models   |=
\newcommand{\lequiv}{\leftrightarrow}
\newcommand{\func}[1]{\textbf{#1}}

% 模型:$\mathcal{M} = (\mathcal{D}, \mathcal{I})$
% 论域:$\mathcal{D}$
% 解释:$\mathcal{I}$
\newcommand{\model}{\mathcal{M}}
\newcommand{\domain}{\mathcal{D}}
\newcommand{\interpretation}{\mathcal{I}}

% 变量赋值:$\assignment: \varset \to \domain$
\let\mapping\rho
\let\assignment\rho
\let\env\mapping

% 语义:$|[\cdot|]$
\newcommand{\sem}[1]{|[#1|]}                        % 程序语义 / 无下标的语义
\newcommand{\plsem}[1]{|[#1|]_{\assignment}}        % 命题逻辑语义
\newcommand{\fosem}[1]{|[#1|]_{\model,\assignment}} % 一阶逻辑语义

% 语义蕴涵和语义等价: $\simply, \sequiv$
\let\simply\entails
\let\sequiv\Leftrightarrow

\begin{document}

\maketitle

\textit{在开始完成作业前,请仔细阅读以下说明:}
\begin{itemize}
    \item 我们提供作业的\LaTeX 源码,你可以在其中直接填充你的答案并编译PDF(请使用xelatex)。
    当然,你也可以使用别的方式完成作业(例如撰写纸质作业后扫描到PDF文件之中)。
    但是请注意,最终的提交一定只是PDF文件。提交时请务必\emph{再次核对},防止提交错误。
    \item 在你的作业中,请务必填写你的\emph{姓名}和\emph{学号},并检查是否有题目遗漏。请重点注意每次作业的\emph{截止时间}。
    截止时间之后你仍可以联系助教补交作业,但是我们会按照如下公式进行分数的折扣:
    \begin{align*}
        \text{作业分数} = \min\left(\text{实际分}, \text{满分}\times\left(1 - 10\%\times\min\left(\lceil\text{迟交周数}\rceil, 10\right)\right)\right).
    \end{align*}
    \item 本次作业为独立作业,\emph{禁止}抄袭等一切不诚信行为。作业中,如果涉及参考资料,请\emph{引用}注明。
\end{itemize}

%% problem begins

\problem{循环}

\subproblem 在扩展IMP语言中,下面两个语句是否语义等价,如果等价请给出证明,否则给出反例。
\begin{itemize}
    \item $?p$
    \item $\textbf{if}(p)\ \textbf{skip}\ \textbf{else}\ ?\textbf{false}$
\end{itemize}
\begin{solution}
    % TODO
\end{solution}


\subproblem \textbf{repeat-until}是另一种常见的循环形式,它的定义如下:
$$\textbf{repeat}\ st\ \textbf{until}(p)\equiv st;(?\neg p;st)^*;?p$$
求证下面的霍尔三元组:
\begin{align*}
    \frac{
        \{\varphi\}\ st\ \{\varphi'\} \;\;\;\;
        \{\varphi'\wedge\neg p\}\ st\ \{\varphi'\}
    }{\{\varphi\}\ \textbf{repeat}\ st\ \textbf{until}(p)\ \{\varphi'\wedge p\}}
\end{align*}
\begin{solution}
    % TODO
\end{solution}

\newpage
\problem{数组}
\subproblem 基于数组理论$\mathcal{T}_A$(及其扩展)编码以下陈述:
\begin{enumerate}
    \item 数组$a$不含有两个相同的元素;
    \item 数组$a$和$b$具有完全相同的元素,则对两者同一位置进行相同更新操作之后,数组$a$和$b$的元素仍然相同。
\end{enumerate}

\begin{solution}
    % TODO
\end{solution}

\subproblem 在扩展IMP语言中,试证明下面的霍尔三元组成立:
\begin{align*}
    &\left\{m < a[0] \wedge i = 0\right\}\ \\
    &\textbf{while}(i < n)\left\{
        \textbf{if}(m < a[i])\ m := a[i];\ i := i + 1\ \textbf{else}\ \textbf{skip}
    \right\}\ \\
    &\left\{\forall k. \left(0\le k < n \rightarrow m \ge a[k]\right)\right\}
\end{align*}
\begin{solution}
    % TODO
\end{solution}

\end{document}
