%% compile with xelatex
\documentclass[11pt,a4paper]{article}

\usepackage[UTF8]{ctex}
\usepackage{homework}

\title{作业 2}
\duedate{Apr 10, 2022}
% TODO your name and ID
\studentname{周雨豪}
\studentid{2018013399}

\usepackage{tikz}
\usetikzlibrary{automata,shapes,positioning,arrows}

%% logical symbols
% \land     /\
% \lor      \/
% \lnor     (negation)
% \to       ->
% \lequiv   <->
% \models   |=
\newcommand{\lequiv}{\leftrightarrow}
\newcommand{\func}[1]{\textbf{#1}}

% 模型:$\mathcal{M} = (\mathcal{D}, \mathcal{I})$
% 论域:$\mathcal{D}$
% 解释:$\mathcal{I}$
\newcommand{\model}{\mathcal{M}}
\newcommand{\domain}{\mathcal{D}}
\newcommand{\interpretation}{\mathcal{I}}

% 变量赋值:$\assignment: \varset \to \domain$
\let\mapping\rho
\let\assignment\rho
\let\env\mapping

% 语义:$|[\cdot|]$
\newcommand{\sem}[1]{|[#1|]}                        % 程序语义 / 无下标的语义
\newcommand{\plsem}[1]{|[#1|]_{\assignment}}        % 命题逻辑语义
\newcommand{\fosem}[1]{|[#1|]_{\model,\assignment}} % 一阶逻辑语义

% 语义蕴涵和语义等价: $\simply, \sequiv$
\let\simply\entails
\let\sequiv\Leftrightarrow

\begin{document}

\maketitle

\textit{在开始完成作业前,请仔细阅读以下说明:}
\begin{itemize}
    \item 我们提供作业的\LaTeX 源码,你可以在其中直接填充你的答案并编译PDF(请使用xelatex)。
    当然,你也可以使用别的方式完成作业(例如撰写纸质作业后扫描到PDF文件之中)。
    但是请注意,最终的提交一定只是PDF文件。提交时请务必\emph{再次核对},防止提交错误。
    \item 在你的作业中,请务必填写你的\emph{姓名}和\emph{学号},并检查是否有题目遗漏。请重点注意每次作业的\emph{截止时间}。
    截止时间之后你仍可以联系助教补交作业,但是我们会按照如下公式进行分数的折扣:
    \begin{align*}
        \text{作业分数} = \min\left(\text{实际分}, \text{满分}\times\left(1 - 10\%\times\min\left(\lceil\text{迟交周数}\rceil, 10\right)\right)\right).
    \end{align*}
    \item 本次作业为独立作业,\emph{禁止}抄袭等一切不诚信行为。作业中,如果涉及参考资料,请\emph{引用}注明。
\end{itemize}

%% problem begins

\problem{一阶理论}

\subproblem 请基于皮亚诺算术理论$\mathcal{T}_{PA}$表示如下的公式:
$$\forall x, y.\ \exists z.\ 2x + 1 < 3y + z$$
\begin{solution}
    % TODO
    $\forall x,y.\exists z,w.2x+1+w=3y+z$
\end{solution}

\subproblem 证明$\mathcal{T}_{\mathbb{Z}}$可以归约到$\mathcal{T}_{\mathbb{N}}$。
\begin{solution}
    % TODO
    签名$\Sigma_{\mathbb{Z}}$中的整数部分归约为$\exists z.z=x+1+...+1$或$\exists z.z+1+...+1=x$,$x$为签名$\Sigma_{\mathbb{N}}$中的整数。数乘$x\cdot y$规约为$x$个$y$相加。算数减号$-x$归约为$+(-x)$。大于号$x>y$规约为$\exists z.x=y+z$。
\end{solution}

\newpage
\problem{程序语义}
\subproblem 计算下列表达式在给定状态$\left\{x\mapsto 3, y\mapsto 1, z\mapsto 0\right\}$下的值:
\begin{enumerate}
    \item $x \le y \rightarrow z \neq x$;
    \item $(x-z)*(y+1)$。
\end{enumerate}

\begin{solution}
	\begin{enumerate}
    	\item $3 \le 1 \rightarrow 0 \neq 3\Leftrightarrow true$;
    	\item $(3-0)*(1+1)=3*2=6$。
	\end{enumerate}
\end{solution}

\subproblem 证明IMP程序语句$\textbf{while}(x<0)\ \{x := y * y;\}$和$\textbf{if}(x<0)\ \{x := y * y;\}\textbf{\ else\ skip}$是语义等价的。

\begin{solution}
    % TODO
    $\llbracket{\textbf{while}(x<0)\ \{x := y * y;\}}\rrbracket=\{(s,s')\ |$存在一个整数$n$和一组状态序列$s,t_1,t_2,\dots,t_n=s'$,使得对任意的$0\leq i < n:\llbracket{x<0}\rrbracket_{t_i}=true,(t_i,t_{i+1})=\llbracket{x:=y*y}\rrbracket,\llbracket{x<0}\rrbracket=false\}$。\\
    因为$y*y$必不小于0,所以$\llbracket{\textbf{while}(x<0)\ \{x := y * y;\}}\rrbracket=\{(s,s')\ |\ x<0\Leftrightarrow true$且$(s,s')\in \llbracket{x:=y*y}\rrbracket_s$或$x\geq0\Leftrightarrow true$且$(s,s')\in\llbracket{skip}\rrbracket_s=\llbracket{\textbf{if}(x<0)\ \{x := y * y;\}\textbf{\ else\ skip}}\rrbracket$\\
    由此两者语义等价
\end{solution}

\newpage
\problem{Hoare逻辑}

\subproblem 试证明如果霍尔三元组
$\left\{\varphi\right\}\ \textbf{if}(p)\left\{st_1\right\} \textbf{else} \left\{st_2\right\}\ \left\{\psi\right\}$是有效式,
则霍尔三元组$\left\{\varphi\wedge p\right\}\ st_1\ \left\{\psi\right\}$和$\left\{\varphi\wedge \neg p\right\}\ st_2\ \left\{\psi\right\}$都是有效式。
\begin{solution}
	\begin{align*}
		\left\{\varphi\wedge p\right\}\ st_1\ \left\{\psi\right\}\ \left\{\varphi\wedge \neg p\right\}\ st_2\ \left\{\psi\right\}& \\
		(\text{分支})\ \ \overline{\left\{\varphi\right\}\ \textbf{if}(p)\ \left\{st_1\right\}\ \textbf{else}\ \left\{st_2\right\}\ \left\{\psi\right\}}&
	\end{align*}
	$\models \left\{\varphi\wedge p\right\}\ st_1\ \left\{\psi\right\}\ \left\{\varphi\wedge \neg p\right\}\ st_2\ \left\{\psi\right\}$,则$\left\{\varphi\wedge p\right\}\ st_1\ \left\{\psi\right\}$和$\left\{\varphi\wedge \neg p\right\}\ st_2\ \left\{\psi\right\}$都是有效式。\\
\end{solution}


\subproblem 试证明下面的霍尔三元组成立:
$$\left\{\text{even}(x)\right\}\ \textbf{while}(x>0)\left\{x:=x-1\right\}\ \left\{\text{even}(x)\wedge x\le 0\right\}$$
其中,$\text{even}(x) \leftrightarrow \exists t.\;x = 2t$。
\begin{solution}
    % TODO
\end{solution}
	\begin{align*}
		(\text{赋值})&\overline{\left\{ x-1>0\right\}\ x:=x-1\ \left\{x>0\right\}}& \\
		(\text{结论弱化})&\overline{\left\{ x-1>0\right\}\ x:=x-1\ \left\{\text{even}(x)\right\}}& \\
		(\text{前提加强})&\overline{\left\{\text{even}(x)\wedge x> 0\right\}\ x:=x-1\ \left\{\text{even}(x)\right\}}& \\
		(\text{循环})&\overline{\left\{\text{even}(x)\right\}\ \textbf{while}(x>0)\left\{x:=x-1\right\}\ \left\{\text{even}(x)\wedge x\le 0\right\}}&
	\end{align*}
\end{document}
