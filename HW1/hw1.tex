%!TEX program = xelatex
\documentclass[11pt,a4paper]{article}

\usepackage[UTF8]{ctex}
\usepackage{homework}
\usepackage{amsmath}

\title{作业 1}
\duedate{Mar 20, 2022}
% TODO your name and ID
\studentname{周雨豪}
\studentid{2018013399}

\usepackage{tikz}
\usetikzlibrary{automata,shapes,positioning,arrows}

%% logical symbols
% \land     /\
% \lor      \/
% \lnor     (negation)
% \to       ->
% \lequiv   <->
% \models   |=
\newcommand{\lequiv}{\leftrightarrow}
\newcommand{\func}[1]{\textbf{#1}}

% 模型:$\mathcal{M} = (\mathcal{D}, \mathcal{I})$
% 论域:$\mathcal{D}$
% 解释:$\mathcal{I}$
\newcommand{\model}{\mathcal{M}}
\newcommand{\domain}{\mathcal{D}}
\newcommand{\interpretation}{\mathcal{I}}

% 变量赋值:$\assignment: \varset \to \domain$
\let\mapping\rho
\let\assignment\rho
\let\env\mapping

% 语义:$|[\cdot|]$
\newcommand{\sem}[1]{|[#1|]}                        % 程序语义 / 无下标的语义
\newcommand{\plsem}[1]{|[#1|]_{\assignment}}        % 命题逻辑语义
\newcommand{\fosem}[1]{|[#1|]_{\model,\assignment}} % 一阶逻辑语义

% 语义蕴涵和语义等价: $\simply, \sequiv$
\let\simply\entails
\let\sequiv\Leftrightarrow

\begin{document}

\maketitle

\textit{在开始完成作业前,请仔细阅读以下说明:}
\begin{itemize}
    \item 我们提供作业的\LaTeX 源码,你可以在其中直接填充你的答案并编译PDF(请使用xelatex)。
    当然,你也可以使用别的方式完成作业(例如撰写纸质作业后扫描到PDF文件之中)。
    但是请注意,最终的提交一定只是PDF文件。提交时请务必\emph{再次核对},防止提交错误。
    \item 在你的作业中,请务必填写你的\emph{姓名}和\emph{学号},并检查是否有题目遗漏。请重点注意每次作业的\emph{截止时间}。
    截止时间之后你仍可以联系助教补交作业,但是我们会按照如下公式进行分数的折扣:
    \begin{align*}
        \text{作业分数} = \min\left(\text{实际分}, \text{满分}\times\left(1 - 10\%\times\min\left(\lceil\text{迟交周数}\rceil, 10\right)\right)\right).
    \end{align*}
    \item 本次作业为独立作业,\emph{禁止}抄袭等一切不诚信行为。作业中,如果涉及参考资料,请\emph{引用}注明。
\end{itemize}

%% problem begins

\problem{判断题}

给定下列陈述,请判断其是否正确。如果错误,请给出反例或解释原因。

\subproblem 给定任意的命题逻辑公式,它是否为有效式一定是可判定的。

\begin{solution}
    True
\end{solution}

\subproblem 给定命题逻辑公式$F$和$G$,如果$F$是有效的且$G$不是有效的,
则$F\rightarrow G$一定不可满足。

\begin{solution}
    False. 当G取值为True时候$F\rightarrow G$取值为True,所以并非不可满足
\end{solution}

\subproblem 给定命题逻辑公式$F$和$G$,如果$F$是可满足的且$\neg G$是不可满足的,则$F\wedge G$一定可满足。

\begin{solution}
    True
\end{solution}

\subproblem 任意给定一个一阶逻辑公式,一定可以在有限时间内判定其是否有效。

\begin{solution}
    False. 如果公式非有效式则无法在有限时间内判定
\end{solution}

\newpage
\problem{解答题}
\subproblem 考虑下列公式:
$$(P \to (Q \to R)) \to (\lnot R \to (\lnot Q \to \lnot P))$$
请列出它的真值表,并判断:1)它是否有效;2)它是否可满足。

\begin{solution}
    % TODO
	$$
		\begin{tabular}{c|c|c|c|c|c}
    	P & Q & R & $P\rightarrow (Q\rightarrow R))$ &
    	$\lnot R\rightarrow (\lnot Q \rightarrow \lnot P)$ & 原式 \\
    	\hline
    	0 & 0 & 0 & 1 & 1 & 1 \\
    	0 & 0 & 1 & 1 & 1 & 1 \\
   		0 & 1 & 0 & 1 & 1 & 1 \\
   		0 & 1 & 1 & 1 & 1 & 1 \\
   		1 & 0 & 0 & 1 & 0 & 0 \\
    	1 & 0 & 1 & 1 & 1 & 1 \\
    	1 & 1 & 0 & 0 & 1 & 1 \\
    	1 & 1 & 1 & 1 & 1 & 1 \\
  		\end{tabular}
	$$
	1)非有效;2)可满足 \\
\end{solution}

\subproblem 在下列公式中,请标记出所有变元的自由出现:
\begin{equation*}
    \forall x. (f(x) \land \exists y.g(x, y, z)) \land \exists z. g(x, y, z)
\end{equation*}

\begin{solution}
	\begin{equation*}
    	\forall x. (f(x) \land \exists y.g(x, y, \underline{z})) \land \exists z. g(\underline{x}, \underline{y}, z)
	\end{equation*}
	见下划线字母 \\
\end{solution}

\subproblem 考虑论域$\domain = \{\circ, \bullet\}$以及下面的解释函数
\begin{itemize}
  \item $\interpretation(f) = \{(\circ, \circ)\mapsto \circ, (\circ, \bullet)\mapsto \bullet, (\bullet, \circ)\mapsto \bullet, (\bullet, \bullet)\mapsto \bullet\}$
  \item $\interpretation(g) = \{\circ \mapsto \bullet, \bullet \mapsto \circ\}$
  \item $\interpretation(p) = \{(\bullet, \circ), (\bullet, \bullet)\}$
\end{itemize}
求公式$\forall x.p(f(g(x), x), x)$的取值。

\begin{solution}
	\begin{align*}
		&x\mapsto \circ \\
		&\llbracket g(x)\rrbracket=\interpretation(g)(\llbracket x\rrbracket)=\interpretation(g)(\circ)=\bullet \\
		&\llbracket f(g(x),x)\rrbracket=\interpretation(f)(\llbracket g(x)\rrbracket,\llbracket x\rrbracket)=\interpretation(f)(\bullet,\circ)=\bullet \\
		&\llbracket p(f(g(x),x),x)\rrbracket=\interpretation(p)(\bullet,\circ)=true \\
    	&x\mapsto \bullet \\
    	&\llbracket g(x)\rrbracket=\interpretation(g)(\llbracket x\rrbracket)=\interpretation(g)(\bullet)=\circ \\
		&\llbracket f(g(x),x)\rrbracket=\interpretation(f)(\llbracket g(x)\rrbracket,\llbracket x\rrbracket)=\interpretation(f)(\circ,\bullet)=\bullet \\
		&\llbracket p(f(g(x),x),x)\rrbracket=\interpretation(p)(\bullet,\bullet)=true
	\end{align*}
	所以$\forall x.p(f(g(x),x))$取值为true \\
\end{solution}

\subproblem 请使用课程教授的相继式演算系统(包含命题逻辑中的10条规则和4条量词消去规则)构建推导树证明下列两个相继式:
\begin{enumerate}
    \item $\exists x.(p(x)\rightarrow q(x)) \vdash \forall y.p(y) \rightarrow \exists z.q(z)$
    \item $\forall y.p(y)\rightarrow \exists z.q(z) \vdash \exists x.(p(x)\rightarrow q(x))$
\end{enumerate}

\begin{solution}
    \begin{align*}
    1.& \\
	&\overline{p(c) \vdash p(c), q(c)} \	 \  \overline{p(c),q(c) \vdash q(c)} &\mbox{切}\\
	&\overline{p(c)\rightarrow q(c), p(c) \vdash q(c)} &\mbox{左蕴含}\\
	&\overline{p(c)\rightarrow q(c) \vdash \neg p(c), q(c)} &\mbox{右否定}\\
	&\overline{p(c)\rightarrow q(c) \vdash \neg p(c), \exists z.q(z)} &\mbox{右存在}\\
	&\overline{p(c)\rightarrow q(c), p(c) \vdash \exists z.q(z)} &\mbox{左否定}\\
	&\overline{p(c)\rightarrow q(c), \forall y.p(y) \vdash \exists z.q(z)} &\mbox{左全称}\\
	&\overline{\exists x.(p(x)\rightarrow q(x)), \forall y.p(y) \vdash \exists z.q(z)} &\mbox{左存在}\\
	&\overline{\exists x.(p(x)\rightarrow q(x)) \vdash \forall y.p(y) \exists z.q(z)} &\mbox{右蕴含}\\
	2.& \\
	&\overline{p(z)\rightarrow q(z)\vdash p(z)\rightarrow q(z)} &\mbox{切} \\
	&\overline{p(z)\rightarrow \exists z.q(z)\vdash p(z)\rightarrow q(z)} &\mbox{左存在} \\
	&\overline{\forall y.p(y)\rightarrow \exists z.q(z)\vdash p(z)\rightarrow q(z)} &\mbox{左全称} \\
	&\overline{\forall y.p(y)\rightarrow \exists z.q(z)\vdash \exists x.(p(x)\rightarrow q(x))} &\mbox{右存在} \\
	\end{align*}
\end{solution}

\end{document}
